% Portada
% --------------------------------------------------------------


\begin{titlepage}
  \thispagestyle{empty}

  % \noindent\hspace*{\centeroffset}

  \centering
  \includegraphics[width=0.8\textwidth]{pic/logo-ugr.png}\\[1cm]

  \textsc{\Large Proyecto de Fin de Carrera\\%[0.2cm]
  }
  \textsc{Ingeniería en Informática}\\[1cm]
  
  {\Huge\bfseries\sffamily GNU Psychosynth}
  \noindent\rule[-1ex]{\textwidth}{3pt}\\[2.5ex]
  {\large\bfseries 
    A framework for modular, interactive and collaborative sound
    synthesis and live music performance}

  \vspace{1cm}

  \begin{center}
    \textbf{Autor}\\ {Juan Pedro Bolívar Puente}\\[.5cm]
    \textbf{Director}\\
    {Joaquín Fernández-Valdivia}\\[0.5cm]

  \end{center}

  \vfill

  \begin{center}\small

    \includegraphics[width=0.15\textwidth]{pic/logo-decsai.png}\\%[0.1cm]
    \textsc{Departamento de Ciencias de la Computación e Inteligencia Artificial}\\
    \textsc{-----------------------------------}\\
    Granada, Junio de 2011
  \end{center}

  % \addtolength{\textwidth}{\centeroffset}
  % \vspace{\stretch{2}}

\end{titlepage}

%%% Local Variables: 
%%% mode: latex
%%% TeX-master: "00-main"
%%% End: 


% Portada Interior
% --------------------------------------------------------------

\thispagestyle{empty}
\cleardoublepage
\thispagestyle{empty}

\begin{titlepage}
  
  \setlength{\centeroffset}{-0.5\oddsidemargin}
  \addtolength{\centeroffset}{0.5\evensidemargin}
  \thispagestyle{empty}

  %\noindent\hspace*{\centeroffset}

  \begin{minipage}{\textwidth}
    \centering
    \vspace{3.3cm}

    \includegraphics[width=.5\textwidth]{pic/logo-psynth.png} 
    \vspace{0.5cm}

    {\Huge\bfseries\sffamily GNU Psychosynth\\}
    \noindent\rule[-1ex]{\textwidth}{3pt}\\[3.5ex]
    {\large\bfseries A framework for modular,
      interactive and collaborative sound synthesis and live music
      performance\\[4cm]}
  \end{minipage}

  \vspace{2.5cm}
  %\noindent\hspace*{\centeroffset}
  
  \begin{minipage}{\textwidth}
    \centering

    \textbf{Autor}\\ {Juan Pedro Bolívar Puente}\\[2.5ex]
    \textbf{Director}\\
    {Joaquín Fernández-Valdivia}\\[2cm]

  \end{minipage}
  \vspace{\stretch{2}}

\end{titlepage}

%%% Local Variables: 
%%% mode: latex
%%% TeX-master: "00-main"
%%% End: 


% Licencia
% --------------------------------------------------------------

\thispagestyle{empty}
\noindent\rule[-1ex]{\textwidth}{2pt}\\[4.5ex]
\vfill

\noindent {\sf Copyright \copyright 2010-2011 Juan Pedro Bolívar
  Puente.}
\vspace{.5cm}

\noindent Permission is granted to copy, distribute and/or modify this
document under the terms of the GNU Free Documentation License,
Version 1.3 or any later version published by the Free Software
Foundation; with no Invariant Sections, no Front-Cover Texts, and no
Back-Cover Texts.  A copy of the license is included in the appendix
entitled "GNU Free Documentation License".

% Abstract en español
% --------------------------------------------------------------

\clearpage
\thispagestyle{empty}
\begin{center}
{\large\bfseries GNU Psychosynth: Un framework para la síntesis de audio modular, interactiva y colaborativa y la ejecución de música en directo.}\\
\end{center}
\begin{center}
\myName
\end{center}
%\vspace{0.7cm}
\noindent{\textbf{Palabras clave}: síntesis modular, interactividad,
  sistemas colaborativos, programación genérica, tiempo real, redes,
  C++0x, software libre, GNU}\\
\vspace{0.7cm}

\noindent{\textbf{Resumen}}\\
En este proyecto de fin de carrera desarrollamos un sistema de
síntesis de audio modular, interactiva y colaborativa orientado a la
interpretación de música en directo. El sistema puede mezclar y
manipular sonidos desde ficheros, una amplia variedad de generadores y
aplicar filtros y efectos de toda clase. La interactividad se consigue
mediante la técnica de \emph{conexionado dinámico}, que permite
alterar la topología del grafo de síntesis con simples movimientos de
los módulos de síntesis, así como prestando especial atención en la
implementación para que cualquier operación sea aplicable en tiempo
real sin generar distorsiones apreciables. La colaboratividad se logra
con un sistema de sincronización en red que permite configurar la
generación del sonido desde varios ordenadores simultaneamente
conectados en red.

En este desarrollo nos centramos en el \emph{framework} que provee el
sistema. Por un lado, desarrollamos una biblioteca  de
procesamiento de audio utilizando los últimos avances en programación
genérica. Además, construimos un entorno de síntesis modular,
jerárquico y desacoplado con novedosas abstracciones de comunicación
entre hebras para facilitar el desarrollo de procesadores digitales de
señales altamente interactivos.

Todo el software desarrollado es libre y sigue una metodología de
desarrollo continua y abierta. Es parte del proyecto GNU.

% Abstract en inglés
% --------------------------------------------------------------

\clearpage
\thispagestyle{empty}
\begin{center}
{\large\bfseries \myTitle}\\
\end{center}
\begin{center}
\myName
\end{center}
\noindent{\textbf{Keywords}: modular synthesis, interactivity,
  collaborative systems, generic programming, real-time, networking,
  C++0x, free software, GNU}\\
\vspace{0.7cm}

\noindent{\textbf{Abstract}}\\
In this mater's thesis project we develop a system for modular,
interactive and collaborative sound synthesis and live music
performance. The system is capable of mixing and manipulating sounds
from files, a wide range of generators and can apply filters of all
kinds. Interactivity is achieved with the \emph{dynamic patching}
technique, that allows altering the topology of the synthesis graph
with simple movements on the synthesis modules, and also taking
special care on the implementation such that any operation can be done
in real-time without introducing noticeable
distortions. Collaborativity is managed through a network based
synchronisation mechanism that enables modifying the sound generation
through several computers simultaneously connected through the
network.

In this development we concentrate on the \emph{framework} that the
system provides. On the one hand, we develop a library for audio
processing using latest advancements in generic programming. Then, we
build an modular synthesis environment that is hierarchical and lowly
coupled with novel inter-thread communication abstractions to ease the
development of highly interactive digital signal processors. 

All this software is free --- as in freedom --- and follows a
continuous and open development methodology. It is part of the GNU
project.

% % Permiso biblioteca
% % --------------------------------------------------------------

% %\chapter*{}
% \cleardoublepage
% \thispagestyle{empty}
% \noindent\rule[-1ex]{\textwidth}{2pt}\\[4.5ex]

% Yo, \textbf{Juan Pedro Bolívar Puente}, alumno de la titulación
% Ingeniería en Informática de la \textbf{Escuela Técnica Superior de
%   Ingenierías Informática y de Telecomunicación de la Universidad de
%   Granada}, con DNI 48941569F, autorizo la ubicación de la siguiente
% copia de mi Proyecto Fin de Carrera en la biblioteca del centro para
% que pueda ser consultada por las personas que lo deseen.

% \vspace{6cm}
% \noindent Fdo: Juan Pedro Bolívar Puente

% \vspace{2cm}

% \begin{flushright}
% Granada a 1 de Julio de 2011.
% \end{flushright}


% % Permiso presentación
% % --------------------------------------------------------------

% \clearpage
% \thispagestyle{empty}
% \noindent\rule[-1ex]{\textwidth}{2pt}\\[4.5ex]

% D. \textbf{Joaquín Fernández-Valdivia}, Catedrático del Departamento
% de Ciencias de la Computación e Inteligencia Artificial de la
% Universidad de Granada.  \vspace{0.5cm}

% \textbf{Informa:}
% \vspace{0.5cm}

% Que el presente proyecto, titulado \textit{\textbf{\myTitle}}, ha sido
% realizado bajo su supervisión por \textbf{Juan Pedro Bolívar Puente},
% y autoriza la defensa de dicho proyecto ante el tribunal que
% corresponda.  \vspace{0.5cm}

% Y para que conste, expide y firma el presente informe en Granada a 7
% de Julio de 2010. 
% \vspace{1cm}

% \textbf{El director:}
% \vspace{4cm}

% \noindent 
% \textbf{D. Joaquín Fernández-Valdivia}% \ \ \ \ \ D. Javier Sánchez Monedero}


% Agreadecimientos
% --------------------------------------------------------------

\thispagestyle{empty}

\chapter*{Agradecimientos}

Decían mis papis que es de bien nacido ser agradecido. Por eso quiero
empezar este proyecto de fin de carrera agradeciéndoselo a ellos, a mi
madre y a mi padre, por (intentar) inculcarme esa cultura del esfuerzo
y curiosidad científica sin los cuales este proyecto habría sido
imposible. También a mi hermana, que introdujo en nuestra casa la
sensibilidad musical y es todo un ejemplo de tesón y lucidez.

Quiero agradecérselo también a mis compañeros de piso y amigos Domingo
y Vílches, por haber sido como una familia durante un curso muy
intenso, y aguantar todo tipo de estridencias musicales a toda
voz. También a María Bla Bla, Carlos, Ana y tantos otros compañeros de
la Red de Estudiantes en Movimiento, porque en su afán e ilusión por
construir un mundo mejor dais sentido a lo que hacemos. A los
compañeros del Hacklab ``Colina Roja'', como Pardi, Pedro, JBC y
muchos más, que demuestran que es posible aproximarse a la tecnología
con un espíritu humanista y contestatario. Especialmente a Javi y a
Fernando, que han contribuido directamente a depurar este documento y
el proyecto en general --- eh! También a Curro, que con sus amplios
conocimientos sobre el kernel de Linux contribuyó muchísimo a
organizar las ideas de los búferes múltiples, y cuya polimatía e
independencia intelectual son un modelo a seguir. Y a María Carrasco,
por todo el ánimo, cariño y comprensión que me ha dado desde los
albores de Psychosynth y aguantar mis tendencias, a veces casi
obsesivas, con el proyecto. También a Marta, que descubriéndome a
György Ligeti me abrió las puertas al placer de la cacofonía, de dónde
algún día nacería este proyecto, y recordarme que lo más importante es
no dejar nunca de imaginar. A Alberto Villegas, que mismamente acaba
de arrojar luz en mis diatribas existenciales y a Sergio, Pina,
Lisa, Gauthier, Frank y tantos otros que me ayudasteis a soportar el
frío de aquellas tierras Finlandesas... A Alberto, Laura y todos
otros onubenses que también andan en el exilio y tanto me habéis
enseñado.

Por supuesto, le estoy enormemente agradecido también a Joaquín,
supervisor de este proyecto, por la confianza casi desmedida que pone
en mi y toda la ayuda que lleva ofreciéndome desde que nos
conociéramos en aquella asignatura de Estructuras de Datos, allá por
primero de carrera. Es un ejemplo de compromiso con la docencia, de
excelencia académica y de consecuencia ética. También a sus
compañeros Javier Martínez Baena y Antonio Garrido Carrillo, por
atender mis correos verborreicos sin que nadie se lo haya pedido.

Se lo agradezco a David, de ArtQuimia, por aquellas discusiones tan
educativas sobre la producción musical y contribuir a definir los
requisitos del proyecto. Y a su pupilo Nacho a.k.a. Shaker Zero Ocho,
por haber diseñado los \emph{loops} y \emph{samples} que se
distribuyen con el programa, haber participado en las primeras
exhibiciones públicas del software y haberme enseñado mucho sobre
producción musical --- a parte de ser un músico ecléctico y genial. Y
a Aleksander Morgado, de GNU, por haber testeado y empaquetado para
Ubuntu y Trisquel cada lanzamiento del software. Y a Miguel
Vázquez-Prada, de Tangiblex, por revisar este documento y ayudarme a
valorar el calado de este proyecto. Y a Lubomir Bourdev, por atender
mis dudas sobre Boost.GIL.

A todos vosotros y muchos más que no estéis por concisión o
desmemoria, pero sin los cuales este proyecto no habría sido posible.

\vspace{1cm}
%\noindent Sinceramente gracias.
De verdad de la buena, $gracias \mapsto \infty$


%%% Local Variables: 
%%% mode: latex
%%% TeX-master: "00-main"
%%% End: 

% Portada
% --------------------------------------------------------------


\begin{titlepage}
  \thispagestyle{empty}

  % \noindent\hspace*{\centeroffset}

  \centering
  \includegraphics[width=0.8\textwidth]{pic/logo-ugr.png}\\[1cm]

  \textsc{\Large Proyecto de Fin de Carrera\\%[0.2cm]
  }
  \textsc{Ingeniería en Informática}\\[1cm]
  
  {\Huge\bfseries\sffamily GNU Psychosynth}
  \noindent\rule[-1ex]{\textwidth}{3pt}\\[2.5ex]
  {\large\bfseries 
    A framework for modular, interactive and collaborative sound
    synthesis and live music performance}

  \vspace{1cm}

  \begin{center}
    \textbf{Autor}\\ {Juan Pedro Bolívar Puente}\\[.5cm]
    \textbf{Director}\\
    {Joaquín Fernández-Valdivia}\\[0.5cm]

  \end{center}

  \vfill

  \begin{center}\small

    \includegraphics[width=0.15\textwidth]{pic/logo-decsai.png}\\%[0.1cm]
    \textsc{Departamento de Ciencias de la Computación e Inteligencia Artificial}\\
    \textsc{-----------------------------------}\\
    Granada, Junio de 2011
  \end{center}

  % \addtolength{\textwidth}{\centeroffset}
  % \vspace{\stretch{2}}

\end{titlepage}

%%% Local Variables: 
%%% mode: latex
%%% TeX-master: "00-main"
%%% End: 


% Portada Interior
% --------------------------------------------------------------

\thispagestyle{empty}
\cleardoublepage
\thispagestyle{empty}

\begin{titlepage}
  
  \setlength{\centeroffset}{-0.5\oddsidemargin}
  \addtolength{\centeroffset}{0.5\evensidemargin}
  \thispagestyle{empty}

  %\noindent\hspace*{\centeroffset}

  \begin{minipage}{\textwidth}
    \centering
    \vspace{3.3cm}

    \includegraphics[width=.5\textwidth]{pic/logo-psynth.png} 
    \vspace{0.5cm}

    {\Huge\bfseries\sffamily GNU Psychosynth\\}
    \noindent\rule[-1ex]{\textwidth}{3pt}\\[3.5ex]
    {\large\bfseries A framework for modular,
      interactive and collaborative sound synthesis and live music
      performance\\[4cm]}
  \end{minipage}

  \vspace{2.5cm}
  %\noindent\hspace*{\centeroffset}
  
  \begin{minipage}{\textwidth}
    \centering

    \textbf{Autor}\\ {Juan Pedro Bolívar Puente}\\[2.5ex]
    \textbf{Director}\\
    {Joaquín Fernández-Valdivia}\\[2cm]

  \end{minipage}
  \vspace{\stretch{2}}

\end{titlepage}

%%% Local Variables: 
%%% mode: latex
%%% TeX-master: "00-main"
%%% End: 


% Licencia
% --------------------------------------------------------------

\thispagestyle{empty}
\noindent\rule[-1ex]{\textwidth}{2pt}\\[4.5ex]
\vfill

\noindent {\sf Copyright \copyright 2010-2011 Juan Pedro Bolívar
  Puente.}
\vspace{.5cm}

\noindent Permission is granted to copy, distribute and/or modify this
document under the terms of the GNU Free Documentation License,
Version 1.3 or any later version published by the Free Software
Foundation; with no Invariant Sections, no Front-Cover Texts, and no
Back-Cover Texts.  A copy of the license is included in the appendix
entitled "GNU Free Documentation License".

% Abstract en español
% --------------------------------------------------------------

\clearpage
\thispagestyle{empty}
\begin{center}
{\large\bfseries GNU Psychosynth: Un framework para la síntesis de audio modular, interactiva y colaborativa y la ejecución de música en directo.}\\
\end{center}
\begin{center}
\myName
\end{center}
%\vspace{0.7cm}
\noindent{\textbf{Palabras clave}: síntesis modular, interactividad,
  sistemas colaborativos, programación genérica, tiempo real, redes,
  C++0x, software libre, GNU}\\
\vspace{0.7cm}

\noindent{\textbf{Resumen}}\\
En este proyecto de fin de carrera desarrollamos un sístema de
síntesis de audio modular, interactiva y colaborativa orientado a la
interpretación de música en directo. El sistema puede mezclar y
manipular sonidos desde ficheros, una amplia variedad de generadores y
aplicar filtros y efectos de toda clase. La interactividad se consigue
mediante la técnica de \emph{conexionado dinámico}, que permite
alterar la topología del grafo de síntesis con simples movimientos de
los módulos de síntesis, así como prestando especial atención en la
implementación para que cualquier operación sea aplicable en tiempo
real sin generar distorsiones apreciables. La colaboratividad se logra
con un sistema de sincronización en red que permite configurar la
generación del sonido desde varios ordenadores simultaneamente
conectados en red.

En este desarrollo nos centramos en el \emph{framework} que provee el
sistema. Por un lado, desarrollamos una bilioteca  de
procesamiento de audio utilizando los últimos avances en programación
genérica. Además, construimos un entorno de síntesis modular,
jerárquico y desacoplado con novedosas abstracciones de comunicación
entre hebras para facilitar el desarrollo de procesadores digitales de
señales altamente interactivos.

Todo el software desarrollado es libre y sigue una metodología de
desarrollo continua y abierta. Es parte del proyecto GNU.

% Abstract en inglés
% --------------------------------------------------------------

\clearpage
\thispagestyle{empty}
\begin{center}
{\large\bfseries \myTitle}\\
\end{center}
\begin{center}
\myName
\end{center}
\noindent{\textbf{Keywords}: modular synthesis, interactivity,
  collaborative systems, generic programming, real-time, networking,
  C++0x, free software, GNU}\\
\vspace{0.7cm}

\noindent{\textbf{Abstract}}\\
In this mater's thesis project we develop a system for modular,
interactive and collaborative sound synthesis and live music
performance. The system is capable of mixing and manipulating sounds
from files, a wide range of generators and can apply filters of all
kinds. Interactivity is achieved with the \emph{dynamic patching}
technique, that allows altering the topology of the synthesis graph
with simple movements on the synthesis modules, and also taking
special care on the implementation such that any operation can be done
in real-time without introducing noticeable
distortions. Collaborativity is managed through a network based
synchronisation mechanism that enables modifying the sound generation
through several computers simultaneously connected through the
network.

In this development we concentrate on the \emph{framework} that the
system provides. On the one hand, we develop a library for audio
processing using latest advancements in generic programming. Then, we
build an modular synthesis environment that is hierarchical and lowly
coupled with novel inter-thread communication abstractions to ease the
development of highly interactive digital signal processors. 

All this software is free --- as in freedom --- and follows a
continuous and open development methodology. It is part of the GNU
project.

% Permiso biblioteca
% --------------------------------------------------------------

%\chapter*{}
\cleardoublepage
\thispagestyle{empty}
\noindent\rule[-1ex]{\textwidth}{2pt}\\[4.5ex]

Yo, \textbf{Juan Pedro Bolívar Puente}, alumno de la titulación
Ingeniería en Informática de la \textbf{Escuela Técnica Superior de
  Ingenierías Informática y de Telecomunicación de la Universidad de
  Granada}, con DNI 48941569F, autorizo la ubicación de la siguiente
copia de mi Proyecto Fin de Carrera en la biblioteca del centro para
que pueda ser consultada por las personas que lo deseen.

\vspace{6cm}
\noindent Fdo: Juan Pedro Bolívar Puente

\vspace{2cm}

\begin{flushright}
Granada a 1 de Junio de 2011.
\end{flushright}


% Permiso presentación
% --------------------------------------------------------------

\clearpage
\thispagestyle{empty}
\noindent\rule[-1ex]{\textwidth}{2pt}\\[4.5ex]

D. \textbf{Joaquín Fernández-Valdivia}, Catedrático del Departamento
de Ciencias de la Computación e Inteligencia Artificial de la
Universidad de Granada.  \vspace{0.5cm}

\textbf{Informa:}
\vspace{0.5cm}

Que el presente proyecto, titulado \textit{\textbf{\myTitle}}, ha sido
realizado bajo su supervisión por \textbf{Juan Pedro Bolívar Puente},
y autoriza la defensa de dicho proyecto ante el tribunal que
corresponda.  \vspace{0.5cm}

Y para que conste, expide y firma el presente informe en Granada a 7
de Julio de 2010.
\vspace{1cm}

\textbf{El director:}
\vspace{4cm}

\noindent 
\textbf{D. Joaquín Fernández-Valdivia}% \ \ \ \ \ D. Javier Sánchez Monedero}


% Agreadecimientos
% --------------------------------------------------------------

\chapter*{Agradecimientos}
\thispagestyle{empty}
\vspace{1cm}
\todo{Añadir agradecimientos.}

% Fernando ---
% David de artquimia por Reaktor
% Shaker08
% Alexander de ubuntu

\vspace{3cm}
%\noindent Sinceramente gracias.

%%% Local Variables: 
%%% mode: latex
%%% TeX-master: "00-main"
%%% End: 
